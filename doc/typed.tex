\documentclass[main.tex]{subfiles}
\begin{document}

\section{\typedlambdalvar}
\usingnamespace{lvar}

\typedlambdalvar is defined over an abstract domain $\mathcal{D}$ which forms a lattice. We write $\botD$ and $\topD$ for the minimum and maximum elements, $\sqsubseteqD$ for its partial order, and $\sqcupD$ for the maximum.

\paragraph*{Terms}
Terms ($\tm{L}$, $\tm{M}$, $\tm{N}$) are defined by the following grammar:
\[
  \begin{array}{lrll}
  \tm{L}, \tm{M}, \tm{N}
  & \coloneqq & \tm{x}
    \sep        \tm{\lambda x.M}
    \sep        \tm{M\;N}                     & \text{functions} \\
  & \sep      & \tm{l}                        
    \sep        \tm{d}
    \sep        \tmJ
    \sep        \tm{K}                        & \text{constants} \\
  & \sep      & \tm{\unit}                    
    \sep        \tm{\letunit{M}{N}}           & \text{units} \\
  & \sep      & \tm{\pair{M}{N}}              
    \sep        \tm{\letpair{x}{y}{M}{N}}     & \text{pairs}
  \\
  \\
  \tm{K}
  & \coloneqq & \tm{\lvarnew}
    \sep        \tm{\lvarfreeze}
    \sep        \tm{\lvarget}
    \sep        \tm{\lvarput}
\end{array}
\]
Let $\tm{x}$, $\tm{y}$, and $\tm{z}$ range over variable names. The term language is the standard $\lambda$-calculus with products and units, extended with LVars~\citep{kuper15}. Let $\tm{l}$ range over location names. Let $\tm{d}$ range over \emph{domain values} from the domain $\mathcal{D}$. Finally, let $\tmJ$ range over \emph{thresholds}. A~threshold $\tmJ$ is a subset of $\mathcal{D}$. A~threshold is pair-wise incompatible, written $\incomp(\tmJ)$, if for all $\tm{d},\tm{d'}\in\mathcal{D}$, if $\tm{d}\neq\tm{d'}$ then $\tm{d}\sqcup\tm{d'}=\tm{\topD}$.

\paragraph*{Configurations}
Configurations ($\tm{C}$), stores ($\tm{S}$), states ($\tm{p}$), and status bits ($\tm{\frz}$) are defined by the following grammar:
\[
\begin{array}{lrll}
  \tm{C}
  & \coloneqq & \tm{\config{M}{S}} & \text{configurations}
  \\
  \tm{S}
  & \coloneqq & \tm{\storeemp}
    \sep        \tm{\storeext{S}{l}{p}} & \text{stores}
  \\
  \tm{p}
  & \coloneqq & \tm{\langle \frz, d \rangle} & \text{states}
  \\
  \tm{\frz}
  & \coloneqq & \tm{\true}
    \sep        \tm{\false}
                                    & \text{status bits}
  \end{array}
\]
A~configuration $\tm{\config{M}{S}}$ consists of a term $\tm{M}$ and a store $\tm{S}$. A~store is a partial function from locations $\tm{l}$ to states $\tm{p}$. A~state consists of a domain value $\tm{d}$ and a status bit $\tm{\frz}$. The status bit records if that particular location has been frozen. We write $\tm{\storeext{S}{l}{p}}$ for the \emph{extension} of a store $\tm{S}$ with state $\tm{p}$ at a \emph{fresh} location $\tm{l}$. We write $\tm{\storeupd{S}{l}{p}}$ for the \emph{update} of a store with state $\tm{p}$ at an \emph{existing} location $\tm{l}$.

We write $\tm{\frz}\sqcupB\tm{\frz'}$ for the maximum on status bits, which is standard Boolean disjunction, and $\tm{\frz}\sqsubseteqB\tm{\frz'}$ for the \emph{total} order on status bits, which is standard Boolean implication.

\paragraph*{Syntactic sugar}
We use syntactic sugar to make terms more readable: we write $\tm{\andthen{M}{N}}$ in place of $\tm{\letunit{M}{N}}$, $\tm{\letbind{x}{M}{N}}$ in place of $\tm{(\lambda x.N)\;M}$, and pattern matching functions $\tm{\lambda\unit.M}$ in place of $\tm{\lambda z.\letunit{z}{M}}$ and $\tm{\lambda\pair{x}{y}.M}$ in place of $\tm{\lambda z.\letpair{x}{y}{z}{M}}$.

\paragraph{Operational semantics}
\label{sec:lvar-semantics}
\begin{mathpar}
  \begin{array}{lrll}
    \tm{V}, \tm{W}
    & \coloneqq & \tm{\lambda x.M}
      \sep        \tm{l}
      \sep        \tm{d}
      \sep        \tmJ
      \sep        \tm{K}
      \sep        \tm{\unit}
      \sep        \tm{\pair{V}{W}}
    \\
    \tm{E}
    & \coloneqq & \tm{\hole}
      \sep        \tm{M\; E}
      \sep        \tm{E\; M}
      \sep        \tm{(M, E)}
      \sep        \tm{(E, M)} \\
    & \sep      & \tm{\letunit{E}{M}}
      \sep        \tm{\letpair{x}{y}{E}{M}}
  \end{array}
\end{mathpar}

\begin{figure*}
  \noindent
  \begin{minipage}[t]{.75\textwidth}
    \fbox{$\tm{C}\red\tm{C'}$}\hfill
    \\
    \begin{mathpar}
      \begin{array}{llrl}
        \LabTirName{E-Lam}     & \tm{\config{(\lambda x.M)\;V}{S}}
                               & \red & \tm{\config{\subst{M}{V}{x}}{S}}
        \\
        \LabTirName{E-Unit}    & \tm{\config{\letunit{\unit}{M}}{S}}
                               & \red & \tm{\config{M}{S}}
        \\
        \LabTirName{E-Pair}    & \tm{\config{\letpair{x}{y}{\pair{V}{W}}{M}}{S}}
                               & \red & \tm{\config{\subst{\subst{M}{V}{x}}{W}{y}}{S}}
      \end{array}
      \\
      \inferrule*[lab=E-New]{
        \tm{l}\not\in\dom(\tm{S})
      }{\tm{\config{\lvarnew}{S}}
        \red
        \tm{\config{l}{S, l \mapsto\st{\botD}{\false}}}}

      \inferrule*[lab=E-Get]{
        \tm{S(l)} = \tm{\st{d}{\frz}}
        \\
        \incomp({\tmJ})
        \\
        \tm{d'}\in\tmJ
        \\
        \tm{d'}\sqsubseteqD\tm{d}
      }{\tm{\config{\lvarget\;l\;\tmJ}{S}}
        \red
        \tm{\config{d'}{S}}}

      \inferrule*[lab=E-Put]{
        \tm{S(l)} = \tm{\st{d}{\false}}
        \\
        \tm{{d}\sqcupD{d'}} \neq \tm{\topD}
      }{\tm{\config{\lvarput\;l\;d'}{S}}
        \red
        \tm{\config{()}{\storeupd{S}{l}{\st{d'}{\false}}}}}
            
      \inferrule*[lab=E-Freeze]{
        \tm{S(l)} = \tm{\st{d}{\frz}}
      }{\tm{\config{\lvarfreeze}{S}}
        \red
        \tm{\config{d}{\storeupd{S}{l}{\st{d}{\true}}}}}
    \end{mathpar}
  \end{minipage}%
  \begin{minipage}[t]{.25\textwidth}
    \fbox{$\tm{C}\rede\tm{C'}$}\hfill
    \\
    \begin{mathpar}
      \inferrule*[lab=E-Lift]{
        \tm{\config{M}{S}} \red \tm{\config{N}{S'}}
      }{\tm{\config{\plug{E}{M}}{S}}
        \rede
        \tm{\config{\plug{E}{N}}{S'}}}
    \end{mathpar}
  \end{minipage}
  \caption{Operational semantics for \typedlambdalvar.}
  \label{fig:semantics}
\end{figure*}
See~\cref{fig:semantics}.

\paragraph*{Types}
Types ($\ty{T}$, $\ty{U}$),  and typing environments ($\ty{\Gamma}$) are defined by the following grammar:
\[
\begin{array}{lrll}
  \ty{T}, \ty{U}
  & \coloneqq & \ty{\tyunit}        & \text{units} \\
  & \sep      & \ty{\typrod{T}{U}}  & \text{pairs} \\
  & \sep      & \ty{\tyfun{T}{U}}   & \text{functions}\\
  & \sep      & \ty{\tyD{d}}        & \text{domain values }{\sqsubseteq d}\\
  & \sep      & \ty{\tyL{\frz}{d}}  & \text{location with values }{\sqsubseteq d}\\
  & \sep      & \ty{\tyJ}           & \text{thresholds}
  \\
  \ty{\Gamma}
  & \coloneqq & \ty{\emptyenv}
    \sep        \ty{\ty{\Gamma},\tmty{x}{T}}
                                    & \text{typing environments}
\end{array}
\]
Types $\ty{\tyunit}$, $\ty{\typrod{T}{U}}$, and $\ty{\tyfun{T}{U}}$ are the standard unit, pair, and function types.
The type $\tyD{d}$ types domain values with the upper bound $d$.
The type $\tyL{\frz}{d}$ types locations which store values of type $\tyD{d}$.
Thresholds occur on both the type- and term-level. Each threshold type $\tyJ$ has only one inhabitant, which is the same threshold $\tmJ$ on the term-level.

We extend $\sqcupD$ and $\sqsubseteqD$ to types and environments, see~\cref{fig:lattice}.
The operation $\sqcupT$ takes two types or typing environments with identical structure \emph{up to domain bounds}, and returns the type or typing environment with the pointwise maximum of each domain bound. Notably, \emph{status bits are unaffected by $\sqcupT$}.
The relation $\sqsubseteqT$ takes two types or typing environments with identical structure up to domain bounds \emph{and status bits}, and checks if each domain bound or status bit in the first argument is smaller than the corresponding domain bound or status bit in the second argument.

\begin{figure*}
  \begin{mathpar}
    \setlength{\arraycolsep}{2pt}
    \begin{array}{lclcl}
      \ty{\tyunit} & \sqcup & \ty{\tyunit}
      & = & \ty{\tyunit}
      \\
      \ty{\typrod{T}{U}} & \sqcup & \ty{\typrod{T'}{U'}}
      & = & \typrod{(\ty{T}\sqcup\ty{T'})}{(\ty{U}\sqcup\ty{U'})}
      \\
      \ty{\tyfun{T}{U}} & \sqcup & \ty{\tyfun{T'}{U'}}
      & = & \tyfun{(\ty{T}\sqcup\ty{T'})}{(\ty{U}\sqcup\ty{U'})}
      \\
      \tyJ & \sqcup & \tyJ
      & = & \tyJ
      \\
      \tyD{d} & \sqcup & \tyD{d'}
      & = & \tyD{{d}\sqcup{d'}}
      \\
      \tyL{\frz}{d} & \sqcup & \tyL{\frz}{d'}
      & = & \tyL{\frz}{{d}\sqcup{d'}}
      \\
      \\
      \ty{\emptyenv} & \sqcup & \ty{\emptyenv}
      & = & \ty{\emptyenv}
      \\
      \ty{\Gamma},\tmty{x}{T} & \sqcup & \ty{\Gamma'},\tmty{x}{U}
      & = & \ty{\Gamma}\sqcup\ty{\Gamma'},\tm{x}:{\ty{T}\sqcup\ty{U}}
    \end{array}
  
    \begin{array}{lclcl}
      \ty{\tyunit} & \sqsubseteq & \ty{\tyunit}
      \\
      \ty{\typrod{T}{U}} & \sqsubseteq & \ty{\typrod{T'}{U'}}
      & \text{if} & {\ty{T}\sqsubseteq\ty{T'}}\text{ and }{\ty{U}\sqsubseteq\ty{U'}}
      \\
      \ty{\tyfun{T}{U}} & \sqsubseteq & \ty{\tyfun{T'}{U'}}
      & \text{if} & {\ty{T}\sqsubseteq\ty{T'}}\text{ and }{\ty{U}\sqsubseteq\ty{U'}}
      \\
      \tyJ & \sqsubseteq & \tyJ
      \\
      \tyD{d} & \sqsubseteq & \tyD{d'}
      & \text{if} & \ty{d}\sqsubseteq\ty{d'}
      \\
      \tyL{\frz}{d} & \sqsubseteq & \tyL{\frz'}{d'}
      & \text{if} & {\ty{\frz}\sqsubseteq\ty{\frz'}}\text{ and }{\ty{d}\sqsubseteq\ty{d'}}
      \\
      \\
      \ty{\emptyenv} & \sqsubseteq & \ty{\emptyenv}
      \\
      \ty{\Gamma},\tmty{x}{T} & \sqsubseteq & \ty{\Gamma'},\tmty{x}{U}
      & \text{if} & \ty{\Gamma}\sqsubseteq\ty{\Gamma'}\text{ and }\ty{T}\sqsubseteq\ty{U}
    \end{array}
  \end{mathpar}
  \todo{%
    Actually, should we go under function types in $\sqsubseteq$?
    Should we change order?}
  \caption{Extension of lattice operations to types and typing environments.}
  \label{fig:lattice}
\end{figure*}


\paragraph{Typing rules}%
\label{sec:lvar-typing}
\begin{figure*}
  \begin{mathpar}
    \fbox{$\seq{\ty{\Gamma}}{\ty{\Delta}}{M}{T}$}\hfill
    \\
    \inferrule*[lab=T-Var]{
      \ty{\Gamma}(\tm{x})=\ty{T}
    }{\seq{\ty{\Gamma}}{\ty{\envemp}}{x}{T}}
  
    \inferrule*[lab=T-Abs]{
      \seq{\ty{\Gamma},\tmty{x}{T}}{\ty{\Sigma'}}{M}{U}
    }{\seq{\ty{\Gamma}}{\ty{\envemp}}{\lambda x.M}{\tyfun[\ty{\Sigma'}]{T}{U}}}
  
    \inferrule*[lab=T-App]{
      \seq{\ty{\Gamma}}{\ty{\Sigma}}{M}{\tyfun[\ty{\Sigma''}]{T}{U}}
      \\
      \seq{\ty{\Gamma}}{\ty{\Sigma'}}{N}{T}
      \\
      \ty{\Sigma}\sqcup\ty{\Sigma'}\sqsubseteq\ty{\Sigma''}
    }{\seq{\ty{\Gamma}}{\ty{\Sigma''}}{M\;N}{U}}

    \inferrule*[lab=T-Unit]{
    }{\seq{\ty{\Gamma}}{\ty{\envemp}}{\unit}{\tyunit}}
    
    \inferrule*[lab=T-Domain]{
    }{\seq{\ty{\Gamma}}{\ty{\envemp}}{d}{\tyD{d}}}

    \inferrule*[lab=T-Location]{
    }{\seq{\ty{\Gamma}}{\tmty{\ell}{\tyL{\frz}{l,u}}}{\ell}{\tyL{\frz}{l,u}}}

    \inferrule*[lab=T-Threshold]{
      \incomp({\tyJ})
    }{\seq{\ty{\Gamma}}{\ty{\Sigma}}{\tmJ}{\tyJ}}

    \inferrule*[lab=T-New]{
    }{\seq{\ty{\Gamma}}{\ty{\Sigma}}{\lvarnew}{\tyL{\false}{\bot,\bot}}}

    \inferrule*[lab=T-Freeze]{
      \seq{\ty{\Gamma}}{\ty{\Sigma}}{M}{\tyL{\true}{l,u}}
    }{\seq{\ty{\Gamma}}{\ty{\Sigma}}{\lvarfreeze\;M}{\tyL{\true}{l,u}}}

    \inferrule*[lab=T-Get]{
      \seq{\ty{\Gamma}}{\ty{\Sigma}}{M}{\tyL{\frz}{l,u}}
      \\
      \seq{\ty{\Gamma}}{\ty{\Sigma'}}{N}{\tyJ}
      \\
      \ty{d}\in\tyJ
      \\
      \ty{d}\sqsubseteq\ty{u}
    }{\seq{\ty{\Gamma}}{\ty{\Sigma}\sqcup\ty{\Sigma'}}{\lvarget\;M\;N}{\tyD{d}}}
  
    \inferrule*[lab=T-Put]{
      \seq{\ty{\Gamma}}{\ty{\Sigma}}{M}{\tyL{\false}{l,d}}
      \\
      \seq{\ty{\Gamma}}{\ty{\Sigma'}}{N}{\tyD{d}}
    }{\seq{\ty{\Gamma}}{\ty{\Sigma}\sqcup\ty{\Sigma'}}{\lvarput\;M\;N}{1}}
  \end{mathpar}
  \caption{Typing rules for \lambdalvar terms.}
  \label{fig:typing}
\end{figure*}

See~\cref{fig:typing}.

\paragraph{Subject reduction}
\begin{lemma}[Subject reduction, $\red$]
  \hfill\\
  If $\seq{\ty{\Gamma}}{C}{T}$ and $\tm{C}\red\tm{C'}$, then $\seq{\ty{\Gamma}}{C}{T}$.
\end{lemma}
\begin{theorem}[Subject reduction, $\rede$]
  \hfill\\
  If $\seq{\ty{\Gamma}}{C}{T}$ and $\tm{C}\red\tm{C'}$, then $\seq{\ty{\Gamma}}{C}{T}$.
\end{theorem}
\begin{proof}
  By induction on the structure of $\tm{C}\red\tm{C'}$.
  \begin{case}{\LabTirName{E-Lam}}
    \[
      \tm{\config{(\lambda x.M)\;V}{S}}
      \red
      \tm{\config{\subst{M}{V}{x}}{S}}
    \]
  \end{case}
  \begin{case}{\LabTirName{E-Unit}}
    \[
      \tm{\config{\letunit{\unit}{M}}{S}}
      \red
      \tm{\config{M}{S}}
    \]
  \end{case}
  \begin{case}{\LabTirName{E-Pair}}    
    \[
      \tm{\config{\letpair{x}{y}{\pair{V}{W}}{M}}{S}}
      \red 
      \tm{\config{\subst{\subst{M}{V}{x}}{W}{y}}{S}}
    \]
  \end{case}
  \begin{case}{\LabTirName{E-New}}     
    \[
      \tm{\config{\lvarnew}{S}}
      \red
      \tm{\config{l}{S, l \mapsto\st{\botD}{\false}}}
    \]
  \end{case}
  \begin{case}{\LabTirName{E-Freeze}}  
    \[
      \tm{\config{\lvarfreeze}{S, l \mapsto\st{d}{\frz}}}
      \red
      \tm{\config{d}{S, l \mapsto\st{d}{\true}}}
    \]
  \end{case}
  \begin{case}{\LabTirName{E-Put}}     
    \[
      \tm{\config{\lvarput\;l\;d'}{S, l \mapsto\st{d}{\false}}}
      \red
      \tm{\config{()}{S, l \mapsto\st{d'}{\false}}}
    \]
  \end{case}
  \begin{case}{\LabTirName{E-Get}}     
    \[
      \tm{\config{\lvarget\;l\;J}{S, l \mapsto\st{d}{\frz}}}
      \red
      \tm{\config{d'}{S, l \mapsto\st{d}{\frz}}}
    \]
  \end{case}
\end{proof}


\paragraph{Progress}
\begin{theorem}[Progress, $\red$]
\end{theorem}
\begin{theorem}[Progress, $\rede$]
\end{theorem}

\end{document}

%%% Local Variables:
%%% TeX-master: "main"
%%% End:
