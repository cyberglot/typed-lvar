\documentclass[main.tex]{subfiles}
\begin{document}

\section{\typedlambdalvar calculus}
\usingnamespace{lvar}

\subsection{Syntax}%
\typedlambdalvar is defined over an abstract domain $\mathcal{D}$ which forms a lattice. We write $\botD$ and $\topD$ for the minimum and maximum elements, $\sqsubseteqD$ for its partial order, and $\sqcupD$ for the maximum.

\paragraph*{Terms}
Terms ($\tm{L}$, $\tm{M}$, $\tm{N}$) are defined by the following grammar:
\[
  \begin{array}{lrll}
  \tm{L}, \tm{M}, \tm{N}
  & \coloneqq & \tm{x}
    \sep        \tm{\lambda x.M}
    \sep        \tm{M\;N}                     & \text{functions} \\
  & \sep      & \tm{l}                        
    \sep        \tm{d}
    \sep        \tmJ
    \sep        \tm{K}                        & \text{constants} \\
  & \sep      & \tm{\unit}                    
    \sep        \tm{\letunit{M}{N}}           & \text{units} \\
  & \sep      & \tm{\pair{M}{N}}              
    \sep        \tm{\letpair{x}{y}{M}{N}}     & \text{pairs}
  \\
  \\
  \tm{K}
  & \coloneqq & \tm{\lvarnew}
    \sep        \tm{\lvarfreeze}
    \sep        \tm{\lvarget}
    \sep        \tm{\lvarput}
\end{array}
\]
Let $\tm{x}$, $\tm{y}$, and $\tm{z}$ range over variable names. The term language is the standard $\lambda$-calculus with products and units, extended with LVars~\citep{kuper15}. Let $\tm{l}$ range over location names. Let $\tm{d}$ range over \emph{domain values} from the domain $\mathcal{D}$. Finally, let $\tmJ$ range over \emph{thresholds}.

A~threshold $\mathcal{J}$ is a subset of $\mathcal{D}$ where all elements are pair-wise incompatible, \ie for all $d,d'\in\mathcal{D}$, if ${d}\neq{d'}$ then ${d}\sqcup{d'}={\top}$.
A~domain value $d$ is below a threshold $\mathcal{J}$, written ${d}\sqsubseteqJ\mathcal{J}$, if there exists a ${d'}\in\mathcal{J}$ such that ${d}\sqsubseteqD{d'}$.

\paragraph*{Types}
Types ($\ty{T}$, $\ty{U}$) and typing environments ($\ty{\Gamma}$) are defined by the following grammar:
\[
\begin{array}{lrll}
  \ty{T}, \ty{U}
  & \coloneqq & \ty{\tyunit}        & \text{units} \\
  & \sep      & \ty{\typrod{T}{U}}  & \text{pairs} \\
  & \sep      & \ty{\tyfun{T}{U}}   & \text{functions}\\
  & \sep      & \ty{\tyD{d}}        & \text{domain values }{\sqsubseteq d}\\
  & \sep      & \ty{\tyL{\frz}{d}}  & \text{location with values }{\sqsubseteq d}\\
  & \sep      & \ty{\tyJ}           & \text{thresholds}
  \\
  \\
  \ty{\Gamma}
  & \coloneqq & \ty{\emptyenv}
    \sep        \ty{\ty{\Gamma},\tmty{x}{T}}
                                    & \text{typing environments}
\end{array}
\]
Types $\ty{\tyunit}$, $\ty{\typrod{T}{U}}$, and $\ty{\tyfun{T}{U}}$ are the standard unit, pair, and function types.
The type $\tyD{d}$ types domain values with the upper bound $d$.
The type $\tyL{\frz}{d}$ types locations which store values of type $\tyD{d}$.
Thresholds occur on both the type- and term-level. Each threshold type $\tyJ$ has only one inhabitant, which is the same threshold $\tmJ$ on the term-level.

We extend $\sqcupD$ and $\sqsubseteqD$ to types and environments, see~\cref{fig:lattice}.
The operation $\sqcupT$ takes two types or typing environments with identical structure \emph{up to domain bounds}, and returns the type with the pointwise maximum of each domain bound. Notably, \emph{status bits are unaffected by $\sqcupT$}.
The relation $\sqsubseteqT$ takes two types or typing environments with identical structure up to domain bounds \emph{and status bits}, and checks if each domain bound or status bit in the first argument is smaller than the corresponding domain bound or status bit in the second argument.

\begin{figure*}
  \begin{mathpar}
    \setlength{\arraycolsep}{2pt}
    \begin{array}{lclcl}
      \ty{\tyunit} & \sqcup & \ty{\tyunit}
      & = & \ty{\tyunit}
      \\
      \ty{\typrod{T}{U}} & \sqcup & \ty{\typrod{T'}{U'}}
      & = & \typrod{(\ty{T}\sqcup\ty{T'})}{(\ty{U}\sqcup\ty{U'})}
      \\
      \ty{\tyfun{T}{U}} & \sqcup & \ty{\tyfun{T'}{U'}}
      & = & \tyfun{(\ty{T}\sqcup\ty{T'})}{(\ty{U}\sqcup\ty{U'})}
      \\
      \tyJ & \sqcup & \tyJ
      & = & \tyJ
      \\
      \tyD{d} & \sqcup & \tyD{d'}
      & = & \tyD{{d}\sqcup{d'}}
      \\
      \tyL{\frz}{d} & \sqcup & \tyL{\frz}{d'}
      & = & \tyL{\frz}{{d}\sqcup{d'}}
      \\
      \\
      \ty{\emptyenv} & \sqcup & \ty{\emptyenv}
      & = & \ty{\emptyenv}
      \\
      \ty{\Gamma},\tmty{x}{T} & \sqcup & \ty{\Gamma'},\tmty{x}{U}
      & = & \ty{\Gamma}\sqcup\ty{\Gamma'},\tm{x}:{\ty{T}\sqcup\ty{U}}
    \end{array}
  
    \begin{array}{lclcl}
      \ty{\tyunit} & \sqsubseteq & \ty{\tyunit}
      \\
      \ty{\typrod{T}{U}} & \sqsubseteq & \ty{\typrod{T'}{U'}}
      & \text{if} & {\ty{T}\sqsubseteq\ty{T'}}\text{ and }{\ty{U}\sqsubseteq\ty{U'}}
      \\
      \ty{\tyfun{T}{U}} & \sqsubseteq & \ty{\tyfun{T'}{U'}}
      & \text{if} & {\ty{T}\sqsubseteq\ty{T'}}\text{ and }{\ty{U}\sqsubseteq\ty{U'}}
      \\
      \tyJ & \sqsubseteq & \tyJ
      \\
      \tyD{d} & \sqsubseteq & \tyD{d'}
      & \text{if} & \ty{d}\sqsubseteq\ty{d'}
      \\
      \tyL{\frz}{d} & \sqsubseteq & \tyL{\frz'}{d'}
      & \text{if} & {\ty{\frz}\sqsubseteq\ty{\frz'}}\text{ and }{\ty{d}\sqsubseteq\ty{d'}}
      \\
      \\
      \ty{\emptyenv} & \sqsubseteq & \ty{\emptyenv}
      \\
      \ty{\Gamma},\tmty{x}{T} & \sqsubseteq & \ty{\Gamma'},\tmty{x}{U}
      & \text{if} & \ty{\Gamma}\sqsubseteq\ty{\Gamma'}\text{ and }\ty{T}\sqsubseteq\ty{U}
    \end{array}
  \end{mathpar}
  \todo{%
    Actually, should we go under function types in $\sqsubseteq$?
    Should we change order?}
  \caption{Extension of lattice operations to types and typing environments.}
  \label{fig:lattice}
\end{figure*}


\paragraph*{Syntactic sugar}
We use syntactic sugar to make terms more readable: we write $\tm{\andthen{M}{N}}$ in place of $\tm{\letunit{M}{N}}$, $\tm{\letbind{x}{M}{N}}$ in place of $\tm{(\lambda x.N)\;M}$, and pattern matching functions $\tm{\lambda\unit.M}$ in place of $\tm{\lambda z.\letunit{z}{M}}$ and $\tm{\lambda\pair{x}{y}.M}$ in place of $\tm{\lambda z.\letpair{x}{y}{z}{M}}$.

\paragraph*{Configurations}
\[
\begin{array}{lrll}
  \tm{\frz}
  & \coloneqq & \tm{\true}
    \sep        \tm{\false}
                                    & \text{status bits}
  \\
  \tm{p}
  & \coloneqq & \tm{\langle \frz, d \rangle} & \text{states}
  \\
  \tm{S}
  & \coloneqq & \tm{\emptystore}
    \sep        \tm{S, l \mapsto p} & \text{stores}
  \\
  \tm{C}
  & \coloneqq & \tm{\config{M}{S}} & \text{configurations}
\end{array}
\]

We write $\tm{\frz}\sqcupB\tm{\frz'}$ for the maximum on status bits, which is standard Boolean disjunction.
We write $\tm{s}\sqcups\tm{s'}$ for the pointwise maximum on states and $\tm{S}\sqcupS\tm{S'}$ for the pointwise maximum on stores. The latter is only defined if the structure of $\tm{S}$ and $\tm{S'}$ is identical up to domain values:
\[
\setlength{\arraycolsep}{2pt}
\begin{array}{lclcl}
  \tm{\st{d}{\frz}} & \sqcups & \tm{\st{d'}{\frz'}}
  & = & \tm{\st{d \sqcupD d'}{\frz \sqcupB \frz'}} \\
  \tm{\emptyenv}      & \sqcupS & \tm{\emptyenv}        & = & \tm{\emptyenv} \\
  \tm{S, l \mapsto p} & \sqcupS & \tm{S', l \mapsto p'} & = & \tm{S \sqcupS S', l \mapsto p \sqcups p'}
\end{array}
\]

\subsection{Operational semantics}
\label{sec:lvar-semantics}
\begin{mathpar}
  \begin{array}{lrll}
    \tm{V}, \tm{W}
    & \coloneqq & \tm{\lambda x.M}
      \sep        \tm{l}
      \sep        \tm{d}
      \sep        \tmJ
      \sep        \tm{K}
      \sep        \tm{\unit}
      \sep        \tm{\pair{V}{W}}
    \\
    \tm{E}
    & \coloneqq & \tm{\hole}
      \sep        \tm{M\; E}
      \sep        \tm{E\; M}
      \sep        \tm{(M, E)}
      \sep        \tm{(E, M)} \\
    & \sep      & \tm{\letunit{E}{M}}
      \sep        \tm{\letpair{x}{y}{E}{M}}
  \end{array}
\end{mathpar}
\begin{figure*}
  \noindent
  \begin{minipage}{.6\textwidth}
  \begin{mathpar}
    \begin{array}{lrll}
      \tm{V}, \tm{W}
      & \coloneqq & \tm{\lambda x.M}
        \sep        \tm{l}
        \sep        \tm{d}
        \sep        \tmJ
        \sep        \tm{K}
        \sep        \tm{\unit}
        \sep        \tm{\pair{V}{W}}
      \\
      \tm{E}
      & \coloneqq & \tm{\hole}
        \sep        \tm{V\; E}
        \sep        \tm{E\; V}
        \sep        \tm{(V, E)}
        \sep        \tm{(E, V)} \\
      & \sep      & \tm{\letunit{E}{M}}
        \sep        \tm{\letpair{x}{y}{E}{M}}
    \end{array}

    \begin{array}{llrl}
      \LabTirName{E-Lam}     & \tm{\config{(\lambda x.M)\;V}{S}}
                             & \red & \tm{\config{\subst{M}{V}{x}}{S}}
      \\
      \LabTirName{E-Unit}    & \tm{\config{\letunit{\unit}{M}}{S}}
                             & \red & \tm{\config{M}{S}}
      \\
      \LabTirName{E-Pair}    & \tm{\config{\letpair{x}{y}{\pair{V}{W}}{M}}{S}}
                             & \red & \tm{\config{\subst{\subst{M}{V}{x}}{W}{y}}{S}}
      \\
      \LabTirName{E-New}     & \tm{\config{\lvarnew}{S}}
                             & \red
                             & \tm{\config{l}{S, l \mapsto\st{\botD}{\false}}}
      \\
      \LabTirName{E-Freeze}  & \tm{\config{\lvarfreeze}{S, l \mapsto\st{d}{\frz}}}
                             & \red
                             & \tm{\config{d}{S, l \mapsto\st{d}{\true}}}
      \\
      \LabTirName{E-Put}     & \tm{\config{\lvarput\;l\;d'}{S, l \mapsto\st{d}{\false}}}
                             & \red
                             & \tm{\config{()}{S, l \mapsto\st{d'}{\false}}}
      \\
      \LabTirName{E-Get}     & \tm{\config{\lvarget\;l\;J}{S, l \mapsto\st{d}{\frz}}}
                             & \red
                             & \tm{\config{d'}{S, l \mapsto\st{d}{\frz}}}
    \end{array}
  \end{mathpar}
  \end{minipage}%
  \begin{minipage}{.4\textwidth}
  \begin{mathpar}
    \inferrule*[lab=E-Par-App]{
      \tm{\config{M}{S}} \red \tm{\config{M'}{S'}}
      \\
      \tm{\config{N}{S}} \red  \tm{\config{N'}{S''}}
    }{\tm{\config{M\;N}{S}}
      \red
      \tm{\config{M'\;N'}{S' \sqcup S''}}
    }
    \\ 
    \inferrule*[lab=E-Par-Pair]{
      \tm{\config{M}{S}} \red \tm{\config{M'}{S'}}
      \\
      \tm{\config{N}{S}} \red  \tm{\config{N'}{S''}}
    }{\tm{\config{\pair{M}{N}}{S}}
      \red
      \tm{\config{\pair{M'}{N'}}{S' \sqcup S''}}
    }
    \\
    \inferrule*[lab=E-Lift]{
      \tm{\config{M}{S}} \red \tm{\config{N}{S'}}
    }{\tm{\config{\plug{E}{M}}{S}}
      \rede
      \tm{\config{\plug{E}{N}}{S'}}
    }
  \end{mathpar}
  \end{minipage}
  \caption{Operational semantics for \typedlambdalvar.}
  \label{fig:semantics}
\end{figure*}
See~\cref{fig:semantics}.

Recall that the evaluation contexts for pairs are $\tm{(V, E)}$ and $\tm{(E, V)}$. Consequently, we can only apply \LabTirName{E-Lift} if either of the components of the pair has already reduced to a value. If both components can still reduce, we are forced to apply \LabTirName{E-Par-Pair}, which reduces both sides in parallel. Similarly for function applications.

\subsection{Typing rules}%
\label{sec:lvar-typing}
\begin{figure*}
  \begin{mathpar}
    \fbox{$\seq{\ty{\Gamma}}{\ty{\Delta}}{M}{T}$}\hfill
    \\
    \inferrule*[lab=T-Var]{
      \ty{\Gamma}(\tm{x})=\ty{T}
    }{\seq{\ty{\Gamma}}{\ty{\envemp}}{x}{T}}
  
    \inferrule*[lab=T-Abs]{
      \seq{\ty{\Gamma},\tmty{x}{T}}{\ty{\Sigma'}}{M}{U}
    }{\seq{\ty{\Gamma}}{\ty{\envemp}}{\lambda x.M}{\tyfun[\ty{\Sigma'}]{T}{U}}}
  
    \inferrule*[lab=T-App]{
      \seq{\ty{\Gamma}}{\ty{\Sigma}}{M}{\tyfun[\ty{\Sigma''}]{T}{U}}
      \\
      \seq{\ty{\Gamma}}{\ty{\Sigma'}}{N}{T}
      \\
      \ty{\Sigma}\sqcup\ty{\Sigma'}\sqsubseteq\ty{\Sigma''}
    }{\seq{\ty{\Gamma}}{\ty{\Sigma''}}{M\;N}{U}}

    \inferrule*[lab=T-Unit]{
    }{\seq{\ty{\Gamma}}{\ty{\envemp}}{\unit}{\tyunit}}
    
    \inferrule*[lab=T-Domain]{
    }{\seq{\ty{\Gamma}}{\ty{\envemp}}{d}{\tyD{d}}}

    \inferrule*[lab=T-Source]{
    }{\seq{\ty{\Gamma}}{\tmty{\ell}{\tyL{\frz}{g,\bot}}}{\ell}{\tyL{\frz}{g,\bot}}}

    \inferrule*[lab=T-Sink]{
    }{\seq{\ty{\Gamma}}{\tmty{\ell}{\tyL{\frz}{\bot,p}}}{\ell}{\tyL{\frz}{\bot,p}}}

    \inferrule*[lab=T-Threshold]{
      \incomp({\tyJ})
    }{\seq{\ty{\Gamma}}{\ty{\Sigma}}{\tmJ}{\tyJ}}

    \inferrule*[lab=T-New]{
    }{\seq{\ty{\Gamma}}{\ty{\Sigma}}{\lvarnew}{\tyL{\false}{\bot,\bot}}}

    \inferrule*[lab=T-Freeze]{
      \seq{\ty{\Gamma}}{\ty{\Sigma}}{M}{\tyL{\true}{g,p}}
    }{\seq{\ty{\Gamma}}{\ty{\Sigma}}{\lvarfreeze\;M}{\tyL{\true}{g,p}}}

    \inferrule*[lab=T-Get]{
      \seq{\ty{\Gamma}}{\ty{\Sigma}}{M}{\tyL{\frz}{g,p}}
      \\
      \seq{\ty{\Gamma}}{\ty{\Sigma'}}{N}{\tyJ}
      \\
      \ty{d}\in\tyJ
      \\
      \ty{d}\sqsubseteq\ty{g}
    }{\seq{\ty{\Gamma}}{\ty{\Sigma}\sqcup\ty{\Sigma'}}{\lvarget\;M\;N}{\tyD{d}}}
  
    \inferrule*[lab=T-Put]{
      \seq{\ty{\Gamma}}{\ty{\Sigma}}{M}{\tyL{\false}{g,p}}
      \\
      \seq{\ty{\Gamma}}{\ty{\Sigma'}}{N}{\tyD{p}}
    }{\seq{\ty{\Gamma}}{\ty{\Sigma}\sqcup\ty{\Sigma'}}{\lvarput\;M\;N}{1}}
  \end{mathpar}
  \caption{Typing rules for \lambdalvar terms.}
  \label{fig:typing}
\end{figure*}

See~\cref{fig:typing}.

\section{Metatheory of \typedlambdalvar calculus}

\subsection{Translation to \lambdalvar  from \typedlambdalvar}

\todo{%
  The translation function $\trans{\cdot}$ should have a tight operational correspondence, \ie terms in \typedlambdalvar should reduce in the same number of steps when translated to \lambdalvar, and should not add or remove synchronisation.}

\todo{%
  The translation function below is unfinished. You probably want to replace each occurrence of an $e$-style variable with a recursive call?}

\begin{mathpar}
  \begin{array}{lcll}
    \tm{\trans{\config{\lvarget\;l\;\tmJ}{S}}}
    & = & \config{S}{\lvarget\;l\;P}
    \\ & & \multicolumn{2}{l}{\text{where } p_1 \cong s \text{ and } P \cong \tmJ}
    \\
    \tm{\trans{\config{\lvarput\;l\;d'}{S}}}
    & = & \config{S}{\lvarput_{i}\;l}
    \\ & & \multicolumn{2}{l}{\text{where }u_{p_{i}}\coloneqq\lambda{d_i}.d\sqcup{d_i}}
    \\
    \tm{\trans{\config{\lvarnew}{S}}}
    & = & \config{S}{\lvarnew}
    \\
    \tm{\trans{\config{\lvarfreeze\;l}{S}}}
    & = & \config{S}{\lvarfreeze\;l}
    \\
    \tm{\trans{\config{\lambda x.M}{S}}}
    & = & \config{S}{\lambda x.e}
    \\
    \tm{\trans{\config{M\;N}{S}}}
    & = & \config{S}{e\;e'}
    \\
    \tm{\trans{\config{()}{S}}}
    & = & \config{S}{()}
    \\
    \tm{\trans{\config{\letunit{M}{N}}{S}}}
    & = & \config{S}{\lambda ().e}
    \\
    \tm{\trans{\config{\pair{M}{N}}{S}}}
    & = & \config{S}{(\lambda x. \lambda y.\lambda f. f x y)\;e\;e'}
    \\
    \tm{\trans{\config{\letpair{x}{y}{M}{N}}{S}}}
    & = & \config{S}{e\;(\lambda x.\lambda y. e')}
    \\
    \tm{\trans{\config{M}{S, l \mapsto (0, d)}}}
    & = & \config{S[l \mapsto (d,\tt{false})]}{e}
    \\
    \tm{\trans{\config{M}{S, l \mapsto (1, d)}}}
    & = & \config{S[l \mapsto (d,\tt{true})]}{e}
  \end{array}
\end{mathpar}

\begin{theorem}[Simulation]
  For any well-typed $\tm{C}$:
  \begin{itemize}
  \item 
    if $\tm{C}\red\tm{C'}$, then $\tm{\trans{C}}\lred\tm{\trans{C'}}$;
  \item
    if $\tm{C}\rede\tm{C'}$, then $\tm{\trans{C}}\lrede\tm{\trans{C'}}$.
  \end{itemize}
\end{theorem}

\begin{theorem}[Reflection]
  For any well-typed $\tm{C}$:
  \begin{itemize}
  \item
    if $\tm{\trans{C}}\lred\tm{\sigma'}$, then $\tm{C}\red\tm{C'}$ and $\tm{\trans{C'}}=\tm{\sigma'}$ for some $\tm{C'}$;
  \item
    if $\tm{\trans{C}}\lrede\tm{\sigma'}$, then $\tm{C}\rede\tm{C'}$ and $\tm{\trans{C'}}=\tm{\sigma'}$ for some $\tm{C'}$.
  \end{itemize}
\end{theorem}


\begin{lemma}[Translation, \typedlambdalvar  $\rightsquigarrow$ \lambdalvar]
  For any translation $\zeta$,
  \begin{itemize}
    \item if $C \red C'$ and $\sigma \hookrightarrow \sigma'$ and $\zeta(C) =
      \sigma$ , then $\zeta(C') = \sigma'$;
    \item if $C \rede C'$ and $\sigma \mapsto \sigma'$ and $\zeta(C) =
      \sigma$ , then $\zeta(C') = \sigma'$.
  \end{itemize}
\end{lemma}

\begin{proof}
  By induction on the structure of $C$. All cases are straight-forward, except
  for the introduction and elimination of pairs.

  \begin{case}{%
      $C = \config{\lvarerror}{S}$,
      $\sigma = \config{S}{\lvarerror}$}
    $C$ and $\sigma$ cannot step. Hence, the translation is vacuously valid.
  \end{case}

  \begin{case}{%
      $C = \config{\lvarget\;l\;J}{S}$,
      $\sigma = \config{S}{\lvarget\;l\;P}$}
    Given the operational semantics, $C$ steps to $C' = \config{s'}{S, l \mapsto (b, d)}$. And given $\lambda_{\text{LVar}}$'s operational semantics, $\sigma$ steps to $\sigma' = \config{S}{p_2}$. Applying $\zeta (C')$, we get $\config{S}{p_2}$. \typedlambdalvar maintains the same number of steps and there is no unwanted synchronisation introduced. Hence, this translation is valid.
  \end{case}

  \begin{case}{%
      $C = \config{\lvarput\;l\;d'}{S}$,
      $\sigma = \config{S}{\lvarput_{i}\;l}$}
    Given the operational semantics, $C$ can either error or take a step.
    \begin{subcase}{$C' = \config{s'}{S, l \mapsto s'}$}
      Given $\lambda_{\text{LVar}}$'s operational semantics, $\sigma$ steps to $\sigma' = \config{S}{p_2}$ if $d \sqcup d_{i} \neq \top$, which is exactly the same as applying $\zeta$ to $C'$. \typedlambdalvar maintains the same number of steps and there is no unwanted synchronisation introduced.
    \end{subcase}
    \begin{subcase}{$C' = \lvarerror$}
      Given $\lambda_{\text{LVar}}$'s operational semantics, $\sigma$ steps to $\sigma' = \lvarerror$ if $d \sqcup d_{i} = \top$, which is exactly the same as applying $\zeta$ to $C'$. \typedlambdalvar maintains the same number of steps and there is no unwanted synchronisation introduced. Hence, this translation is valid.
    \end{subcase}
  \end{case}

  \begin{case}{%
      $C = \config{\lvarnew}{S}$,
      $\sigma = \config{S}{\lvarnew}$}
    $C'$ steps to $\config{l}{S, l \mapsto (0, \bot)}$ which is equivalent to $\config{S[l \mapsto (\bot, false)]}{l}$, as showed in the last two cases of the proof. \typedlambdalvar maintains the same number of steps and there is no unwanted synchronisation introduced. Hence, this translation is valid.
  \end{case}

  \begin{case}{%
      $C = \config{\lvarfreeze}{S}$,
      $\sigma = \config{S}{\lvarfreeze}$}
    $C'$ steps to $\config{d}{S, l \mapsto (1, d)}$ which is equivalent to $\config{S[l \mapsto (p, true)]}{p}$, as showed in the last two cases of the proof. \typedlambdalvar maintains the same number of steps and there is no unwanted synchronisation introduced. Hence, this translation is valid.
  \end{case}

  \begin{case}{%
      $C = \config{\lambda x.M}{S}$,
      $\sigma = \config{S}{\lambda x.e}$}
    $C$ and $\sigma$ do not step since lambda abstractions are values. Also, $C$ and $\sigma$ are immediately equivalent up to $\alpha$-equivalence.
  \end{case}

  \begin{case}{%
      $\config{M\;N}{S}$,
      $ \sigma = \config{S}{e\;e'}$}
    The application case is simple, where expressions take one step each in parallel in both languages. \typedlambdalvar maintains the same number of steps and there is no unwanted synchronisation introduced. Hence, this translation is valid.
  \end{case}

  \begin{case}{%
      $ C = \config{()}{S}$,
      $\sigma = \config{S}{()}$}
    $C$ and $\sigma$ do not step since unit is a value. Also, $C$ and $\sigma$ are immediately equivalent.
  \end{case}

  \begin{case}{%
      $ C = \config{\letunit{M}{N}}{S}$,
      $\sigma = \config{S}{(\lambda (). e')  e}$}
    The \lambdalvar calculus does not provide an elimination rule for unit, since it introduces an explicit synchronisation construct. However, such construct is easily defined by forcing $e$ to evaluate before $e'$ via the introduction and elimination a lambda abstraction. \citet{kuper15} informally uses a generalised version of this construct. Both $C$ and $\sigma$ steps to the outermost expression, $N$ and $e'$, respectively. Explicit unit elimination here is used to synchronise and necessary for \texttt{runLVars} to work. No extra steps are taken in the translation, hence, this translation is valid.
  \end{case}

  \begin{case}{%
      $ C = \config{\pair{M}{N}}{S}$,
      $\sigma = \config{S}{(\lambda x. \lambda y.\lambda f. f x y)\;e\;e'}$}
    In \typedlambdalvar, pair components are evaluated in parallel and the next step will be blocked until both components are evaluated to a value. The \lambdalvar calculus does not provide pairs, therefore we encode them using lambda abstractions. In our encoding, a function takes two values and returns function that takes both values. According to the semantics of the \lambdalvar  calculus, when those two expressions are passed to a function, they are evaluated in parallel. Hence, our encoding does not introduce synchronisation, blocking the next step unnecessarily. The translation maintains the same number of steps and, hence, is valid.
  \end{case}

  \begin{case}{%
      $C = \config{\letpair{x}{y}{M}{N}}{S}$,
      $\sigma = \config{S}{e' \;(\lambda x.\lambda y. e)}$}
    In \typedlambdalvar, the elimination rule for pairs require that both components are values, and those are substituted within the next computation by using two fresh variables. Given that we encoded pairs as a function that takes a function with two arguments, we need to create said function in order to eliminate pairs. The eliminating function has to make the values within the pair available to the next computation, in this case $e'$. According to the \lambdalvar calculus, parameters must be fully evaluated before being passed on to a lambda abstraction - fact easily verifiable since \lambdalvar has call-by-value semantics. Therefore, our encoding does not introduce synchronisation, blocking the next step unnecessarily, and maintains the same number of steps. Hence, this translation is valid.
  \end{case}

  \begin{case}{%
      $C = \config{M}{S, l \mapsto (0, d)}$,
      $\sigma = \config{S[l \mapsto (d, \tt{false})]}{e}$}
    The encoding of LVar's writeability status in \typedlambdalvar uses $0$ and $1$,
    while in \lambdalvar, they are encoded as regular booleans. Irregardless of the most
    common representation of booleans as numbers, the LVar should be initialised
    with one status and switch to a different one once frozen, which happens in
    both \lambdalvar and \typedlambdalvar.
  \end{case}

  \begin{case}{%
      $C = \config{M}{S, l \mapsto (1, d)}$,
      $ \sigma = \config{S[l \mapsto (d, \tt{true})]}{e}$}
    Follows an analogous argument as previous case.
  \end{case}

  \remember{Add case for lifted evaluation contexts.}
\end{proof}


\subsection{Determinism}

Given the translation of \typedlambdalvar  into \lambdalvar  is valid, we can infer that \typedlambdalvar  is quasi-deterministic as well. Here, we restate definitions leading to the quasi-determinism proof as stated in \citet{kuper15}.
\remember{Please don't restate the proofs.}

\definition[Permutation]{
  A~permutation function is a function $\pi : \ty{L} \rightarrow \ty{L}$ such that:
\begin{itemize}
\item it is invertible, that is, there is an inverse function $\pi^{-1}i : \ty{L} \rightarrow \ty{L}$ with the property that $\pi(l) = l'$ iff $\pi^{-1}(l') = l$;
\item it is the identity on all but finitely many elements of $\ty{L}$.
\end{itemize}
}

\definition[Permutation of expressions]{
  A~permutation of an expression $M$ is a function $\pi$ defined as follows:
 \todo{find the base case where something happens?}
  \begin{mathpar}
  \begin{array}{lcll}
    \pi(\lvarget\;l\;J)       & = & \lvarget\;\pi(l)\;\pi(J) \\
    \pi(\lvarput\;l\;d')      & = & \lvarput\;\pi(l) \\
    \pi(\lvarnew)                & = & \lvarnew \\
    \pi(\lvarfreeze\;l)       & = & \lvarfreeze\;\pi(l) \\
    \pi(\lambda x.M)         & = & \lambda x.\pi(M) \\
    \pi(M\;N)                      & = & \pi(M) \; \pi(N)\\
    \pi(())                            & = & () \\
    \pi(x)                            & = & x \\
    \pi(\letunit{M}{N})      & = & \letunit{\pi(M)}{\pi(N)} \\
    \pi(\pair{M}{N})           & = &  \pair{\pi(M)}{\pi(N)} \\
    \pi(\letpair{x}{y}{M}{N}) & = & \letpair{x}{y}{\pi(M)}{\pi(N)} \\
  \end{array}\\
\end{mathpar}}

\definition{Permutation of a store}

\definition{Permutation of configurations}

\lemma{Permutability}

\lemma{Internal Determinism}

\lemma{Strong Confluence}

\lemma{Confluence}

\theorem{Quasi-Determinism}

\subsection{Type safety}

\theorem{Progress}

\theorem{Preservation}

\corollary{Type Safety}

\subsection{Fully deterministic programming with LVars}

In this section, we prove that \typedlambdalvar  is deterministic, which is the main
contribution of this work. We proceed by proving that all \lvarerror  states
within the \typedlambdalvar calculus are not typeable given our type rules.

\theorem{Untypeable \lvarerror s}
\corollary{Full Determinism}

\end{document}

%%% Local Variables:
%%% TeX-master: "main"
%%% End:
