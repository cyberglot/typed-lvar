%%%%%%%%%%%%%%%%%%%%%%%%%%%%%%%%%%%%%%%%%
% kaobook
% LaTeX Template
% Version 1.2 (4/1/2020)
%
% This template originates from:
% https://www.LaTeXTemplates.com
%
% For the latest template development version and to make contributions:
% https://github.com/fmarotta/kaobook
%
% Authors:
% Federico Marotta (federicomarotta@mail.com)
% Based on the doctoral thesis of Ken Arroyo Ohori (https://3d.bk.tudelft.nl/ken/en)
% and on the Tufte-LaTeX class.
% Modified for LaTeX Templates by Vel (vel@latextemplates.com)
%
% License:
% CC0 1.0 Universal (see included MANIFEST.md file)
%
%%%%%%%%%%%%%%%%%%%%%%%%%%%%%%%%%%%%%%%%%

%----------------------------------------------------------------------------------------
%	PACKAGES AND OTHER DOCUMENT CONFIGURATIONS
%----------------------------------------------------------------------------------------

\documentclass[
	fontsize=10pt, % Base font size
	twoside=false, % Use different layouts for even and odd pages (in particular, if twoside=true, the margin column will be always on the outside)
	%open=any, % If twoside=true, uncomment this to force new chapters to start on any page, not only on right (odd) pages
	%chapterprefix=true, % Uncomment to use the word "Chapter" before chapter numbers everywhere they appear
	%chapterentrydots=true, % Uncomment to output dots from the chapter name to the page number in the table of contents
	numbers=noenddot, % Comment to output dots after chapter numbers; the most common values for this option are: enddot, noenddot and auto (see the KOMAScript documentation for an in-depth explanation)
	%draft=true, % If uncommented, rulers will be added in the header and footer
	%overfullrule=true, % If uncommented, overly long lines will be marked by a black box; useful for correcting spacing problems
]{kaobook}

% Set the language
\usepackage[utf8]{inputenc}
\usepackage[english]{babel} % Load characters and hyphenation
\usepackage[english=british]{csquotes} % English quotes

% Load packages for testing
\usepackage{blindtext}
%\usepackage{showframe} % Uncomment to show boxes around the text area, margin, header and footer
%\usepackage{showlabels} % Uncomment to output the content of \label commands to the document where they are used

% Load the bibliography package
\usepackage{styles/kaobiblio}
\addbibresource{main.bib} % Bibliography file

% Load mathematical packages for theorems and related environments. NOTE: choose only one between 'mdftheorems' and 'plaintheorems'.
\usepackage{styles/mdftheorems}
\usepackage{styles/plt}
% \usepackage{styles/plaintheorems}

\graphicspath{{/figures/}{images/}} % Paths in which to look for images

\makeindex[columns=3, title=Alphabetical Index, intoc] % Make LaTeX produce the files required to compile the index

% \makeglossaries % Make LaTeX produce the files required to compile the glossary

% \makenomenclature % Make LaTeX produce the files required to compile the nomenclature

% Reset sidenote counter at chapters
%\counterwithin*{sidenote}{chapter}

\usepackage{subfiles}
\newcommand{\onlyinsubfile}[1]{#1}
\newcommand{\notinsubfile}[1]{}
% \usepackage[all]{foreign}

%----------------------------------------------------------------------------------------

\begin{document}

%----------------------------------------------------------------------------------------
%	BOOK INFORMATION
%----------------------------------------------------------------------------------------

% \titlehead{Typed LVar}

\title[Typed Lattice Variables]{Typed Lattice Variables}
\subtitle{Safe Deterministic Parallelism with Mutable Shared State}

\author[April Goncalves \& Wen Kokke]{April Goncalves \& Wen Kokke}

\date{\today}

% \publishers{An Awesome Publisher}

%----------------------------------------------------------------------------------------

\frontmatter % Denotes the start of the pre-document content, uses roman numerals

%----------------------------------------------------------------------------------------
%	OPENING PAGE
%----------------------------------------------------------------------------------------

%\makeatletter
%\extratitle{
%	% In the title page, the title is vspaced by 9.5\baselineskip
%	\vspace*{9\baselineskip}
%	\vspace*{\parskip}
%	\begin{center}
%		% In the title page, \huge is set after the komafont for title
%		\usekomafont{title}\huge\@title
%	\end{center}
%}
%\makeatother

%----------------------------------------------------------------------------------------
%	COPYRIGHT PAGE
%----------------------------------------------------------------------------------------

% \makeatletter
% \uppertitleback{\@titlehead} % Header

% \lowertitleback{
% 	\textbf{Disclaimer}\\
% 	You can edit this page to suit your needs. For instance, here we have a no copyright statement, a colophon and some other information. This page is based on the corresponding page of Ken Arroyo Ohori's thesis, with minimal changes.

% 	\medskip

% 	\textbf{No copyright}\\
% 	\cczero\ This book is released into the public domain using the CC0 code. To the extent possible under law, I waive all copyright and related or neighbouring rights to this work.

% 	To view a copy of the CC0 code, visit: \\\url{http://creativecommons.org/publicdomain/zero/1.0/}

% 	\medskip

% 	\textbf{Colophon} \\
% 	This document was typeset with the help of \href{https://sourceforge.net/projects/koma-script/}{\KOMAScript} and \href{https://www.latex-project.org/}{\LaTeX} using the \href{https://github.com/fmarotta/kaobook/}{kaobook} class.

% 	The source code of this book is available at:\\\url{https://github.com/fmarotta/kaobook}

% 	(You are welcome to contribute!)

% 	\medskip

% 	\textbf{Publisher} \\
% 	First printed in May 2019 by \@publishers
% }
% \makeatother

%----------------------------------------------------------------------------------------
%	DEDICATION
%----------------------------------------------------------------------------------------

% \dedication{
% 	The harmony of the world is made manifest in Form and Number, and the heart and soul and all the poetry of Natural Philosophy are embodied in the concept of mathematical beauty.\\
% 	\flushright -- D'Arcy Wentworth Thompson
% }

%----------------------------------------------------------------------------------------
%	OUTPUT TITLE PAGE AND PREVIOUS
%----------------------------------------------------------------------------------------

% Note that \maketitle outputs the pages before here

% If twoside=false, \uppertitleback and \lowertitleback are not printed
% To overcome this issue, we set twoside=semi just before printing the title pages, and set it back to false just after the title pages
\KOMAoptions{twoside=semi}
\maketitle
\KOMAoptions{twoside=false}

%----------------------------------------------------------------------------------------
%	PREFACE
%----------------------------------------------------------------------------------------

%\input{chapters/preface.tex}

%----------------------------------------------------------------------------------------
%	TABLE OF CONTENTS & LIST OF FIGURES/TABLES
%----------------------------------------------------------------------------------------

\begingroup % Local scope for the following commands

% Define the style for the TOC, LOF, and LOT
%\setstretch{1} % Uncomment to modify line spacing in the ToC
%\hypersetup{linkcolor=blue} % Uncomment to set the colour of links in the ToC
\setlength{\textheight}{23cm} % Manually adjust the height of the ToC pages

% Turn on compatibility mode for the etoc package
\etocstandarddisplaystyle % "toc display" as if etoc was not loaded
\etocstandardlines % toc lines as if etoc was not loaded

% \tableofcontents % Output the table of contents

% \listoffigures % Output the list of figures

% Comment both of the following lines to have the LOF and the LOT on different pages
\let\cleardoublepage\bigskip
\let\clearpage\bigskip

% \listoftables % Output the list of tables

\endgroup

%----------------------------------------------------------------------------------------
%	MAIN BODY
%----------------------------------------------------------------------------------------

\mainmatter % Denotes the start of the main document content, resets page numbering and uses arabic numbers
\setchapterstyle{kao} % Choose the default chapter heading style

\subfile{chapters/introduction.tex}

% \pagelayout{wide} % No margins
% \pagelayout{margin} % Restore margins

\subfile{chapters/lvar.tex}
\subfile{chapters/typed-lvar.tex}
% \input{chapters/references.tex}

% \pagelayout{wide} % No margins
% \addpart{Design and Additional Features}
% \pagelayout{margin} % Restore margins

% \input{chapters/layout.tex}
% \setchapterstyle{kao}
% \setchapterpreamble[u]{\margintoc}
\chapter{Mathematics and Boxes}
\labch{mathematics}

\section{Theorems}

Despite most people complain at the sight of a book full of equations,
mathematics is an important part of many books. Here, we shall
illustrate some of the possibilities. We believe that theorems,
definitions, remarks and examples should be emphasised with a shaded
background; however, the colour should not be to heavy on the eyes, so
we have chosen a sort of light yellow.\sidenote{The boxes are all of the
same colour here, because we did not want our document to look like
\href{https://en.wikipedia.org/wiki/Harlequin}{Harlequin}.}

\begin{definition}
\labdef{openset}
Let $(X, d)$ be a metric space. A subset $U \subset X$ is an open set
if, for any $x \in U$ there exists $r > 0$ such that $B(x, r) \subset
U$. We call the topology associated to d the set $\tau\textsubscript{d}$
of all the open subsets of $(X, d).$
\end{definition}

\refdef{openset} is very important. I am not joking, but I have inserted
this phrase only to show how to reference definitions. The following
statement is repeated over and over in different environments.

\begin{theorem}
A finite intersection of open sets of (X, d) is an open set of (X, d),
i.e $\tau\textsubscript{d}$ is closed under finite intersections. Any
union of open sets of (X, d) is an open set of (X, d).
\end{theorem}

\begin{proposition}
A finite intersection of open sets of (X, d) is an open set of (X, d),
i.e $\tau\textsubscript{d}$ is closed under finite intersections. Any
union of open sets of (X, d) is an open set of (X, d).\marginnote{You can even insert footnotes inside the theorem
	environments; they will be displayed at the bottom of the box.}
\end{proposition}

\begin{lemma}
A finite intersection\footnote{I'm a footnote} of open sets of (X, d) is
an open set of (X, d), i.e $\tau\textsubscript{d}$ is closed under
finite intersections. Any union of open sets of (X, d) is an open set of
(X, d).
\end{lemma}

You can safely ignore the content of the theorems\ldots I assume that if
you are interested in having theorems in your book, you already know
something about the classical way to add them. These example should just
showcase all the things you can do within this class.

\begin{corollary}[Finite Intersection, Countable Union]
A finite intersection of open sets of (X, d) is an open set of (X, d),
i.e $\tau\textsubscript{d}$ is closed under finite intersections. Any
union of open sets of (X, d) is an open set of (X, d).
\end{corollary}

\begin{proof}
The proof is left to the reader as a trivial exercise. Hint: \blindtext
\end{proof}

\begin{definition}
Let $(X, d)$ be a metric space. A subset $U \subset X$ is an open set
if, for any $x \in U$ there exists $r > 0$ such that $B(x, r) \subset
U$. We call the topology associated to d the set $\tau\textsubscript{d}$
of all the open subsets of $(X, d).$\marginnote{
	Here is a random equation, just because we can:
	\begin{equation*}
  x = a_0 + \cfrac{1}{a_1
          + \cfrac{1}{a_2
          + \cfrac{1}{a_3 + \cfrac{1}{a_4} } } }
	\end{equation*}
}
\end{definition}

\begin{example}
Let $(X, d)$ be a metric space. A subset $U \subset X$ is an open set
if, for any $x \in U$ there exists $r > 0$ such that $B(x, r) \subset
U$. We call the topology associated to d the set $\tau\textsubscript{d}$
of all the open subsets of $(X, d).$
\end{example}

\begin{remark}
Let $(X, d)$ be a metric space. A subset $U \subset X$ is an open set
if, for any $x \in U$ there exists $r > 0$ such that $B(x, r) \subset
U$. We call the topology associated to d the set $\tau\textsubscript{d}$
of all the open subsets of $(X, d).$
\end{remark}

As you may have noticed, definitions, example and remarks have
independent counters; theorems, propositions, lemmas and corollaries
share the same counter.

\begin{remark}
Here is how an integral looks like inline: $\int_{a}^{b} x^2 dx$, and
here is the same integral displayed in its own paragraph:
\[\int_{a}^{b} x^2 dx\]
\end{remark}

We provide two files for the theorem styles:
\href{style/plaintheorems.sty}{plaintheorems.sty}, which you should
include if you do not want coloured boxes around theorems; and
\href{style/mdftheorems.sty}{mdftheorems.sty}, which is the one used for
this document.\sidenote{The plain one is not showed, but actually it is
exactly the same as this one, only without the yellow boxes.} Of course,
you will have to edit these files according to your taste and the
general style of the book.

\section[Boxes \& Environments]{Boxes \& Custom Environments
\sidenote[][*1.8]{Notice that in the table of contents and in the
	header, the name of this section is \enquote{Boxes \& Environments};
	we achieved this with the optional argument of the \texttt{section}
	command.}}

Say you want to insert a special section, an optional content or just
something you want to emphasise. We think that nothing works better than
a box in these cases. We used \Package{mdframed} to construct the ones
shown below. You can create and modify such environments by editing the
provided file \href{style/environments.sty}{environments.sty}.

\begin{kaobox}[frametitle=Title of the box]
\blindtext
\end{kaobox}

If you set up a counter, you can even create your own numbered
environment.

\begin{kaocounter}
	\blindtext
\end{kaocounter}

\section{Experiments}

It is possible to wrap marginnotes inside boxes, too. Audacious readers
are encouraged to try their own experiments and let me know the
outcomes.

\marginnote[-2.2cm]{
	\begin{kaobox}[frametitle=title of margin note]
		Margin note inside a kaobox.\\
		(Actually, kaobox inside a marginnote!)
	\end{kaobox}
}

I believe that many other special things are possible with the
\Class{kaobook} class. During its development, I struggled to keep it as
flexible as possible, so that new features could be added without too
great an effort. Therefore, I hope that you can find the optimal way to
express yourselves in writing a book, report or thesis with this class,
and I am eager to see the outcomes of any experiment that you may try.

%\begin{margintable}
	%\captionsetup{type=table,position=above}
	%\begin{kaobox}
		%\caption{caption}
		%\begin{tabular}{ |c|c|c|c| }
			%\hline
			%col1 & col2 & col3 \\
			%\hline
			%\multirow{3}{4em}{Multiple row} & cell2 & cell3 \\ & cell5
			%%& cell6 \\
			%& cell8 & cell9 \\
			%\hline
		%\end{tabular}
	%\end{kaobox}
%\end{margintable}


% \appendix % From here onwards, chapters are numbered with letters, as is the appendix convention

% \pagelayout{wide} % No margins
% \addpart{Appendix}
% \pagelayout{margin} % Restore margins

% \subfile{chapters/dump.tex}

%----------------------------------------------------------------------------------------

\backmatter % Denotes the end of the main document content
\setchapterstyle{plain} % Output plain chapters from this point onwards

%----------------------------------------------------------------------------------------
%	BIBLIOGRAPHY
%----------------------------------------------------------------------------------------

% The bibliography needs to be compiled with biber using your LaTeX editor, or on the command line with 'biber main' from the template directory

\defbibnote{bibnote}{Here are the references in citation order.\par\bigskip} % Prepend this text to the bibliography
\printbibliography[heading=bibintoc, title=Bibliography, prenote=bibnote] % Add the bibliography heading to the ToC, set the title of the bibliography and output the bibliography note

%----------------------------------------------------------------------------------------
%	NOMENCLATURE
%----------------------------------------------------------------------------------------

% The nomenclature needs to be compiled on the command line with 'makeindex main.nlo -s nomencl.ist -o main.nls' from the template directory

% \nomenclature{$c$}{Speed of light in a vacuum inertial frame}
% \nomenclature{$h$}{Planck constant}

% \renewcommand{\nomname}{Notation} % Rename the default 'Nomenclature'
% \renewcommand{\nompreamble}{The next list describes several symbols that will be later used within the body of the document.} % Prepend this text to the nomenclature

% \printnomenclature % Output the nomenclature

%----------------------------------------------------------------------------------------
%	GREEK ALPHABET
% 	Originally from https://gitlab.com/jim.hefferon/linear-algebra
%----------------------------------------------------------------------------------------

% \vspace{1cm}

% {\usekomafont{chapter}Greek Letters with Pronounciation} \\[2ex]
% \begin{center}
% 	\newcommand{\pronounced}[1]{\hspace*{.2em}\small\textit{#1}}
% 	\begin{tabular}{l l @{\hspace*{3em}} l l}
% 		\toprule
% 		Character & Name & Character & Name \\
% 		\midrule
% 		$\alpha$ & alpha \pronounced{AL-fuh} & $\nu$ & nu \pronounced{NEW} \\
% 		$\beta$ & beta \pronounced{BAY-tuh} & $\xi$, $\Xi$ & xi \pronounced{KSIGH} \\
% 		$\gamma$, $\Gamma$ & gamma \pronounced{GAM-muh} & o & omicron \pronounced{OM-uh-CRON} \\
% 		$\delta$, $\Delta$ & delta \pronounced{DEL-tuh} & $\pi$, $\Pi$ & pi \pronounced{PIE} \\
% 		$\epsilon$ & epsilon \pronounced{EP-suh-lon} & $\rho$ & rho \pronounced{ROW} \\
% 		$\zeta$ & zeta \pronounced{ZAY-tuh} & $\sigma$, $\Sigma$ & sigma \pronounced{SIG-muh} \\
% 		$\eta$ & eta \pronounced{AY-tuh} & $\tau$ & tau \pronounced{TOW (as in cow)} \\
% 		$\theta$, $\Theta$ & theta \pronounced{THAY-tuh} & $\upsilon$, $\Upsilon$ & upsilon \pronounced{OOP-suh-LON} \\
% 		$\iota$ & iota \pronounced{eye-OH-tuh} & $\phi$, $\Phi$ & phi \pronounced{FEE, or FI (as in hi)} \\
% 		$\kappa$ & kappa \pronounced{KAP-uh} & $\chi$ & chi \pronounced{KI (as in hi)} \\
% 		$\lambda$, $\Lambda$ & lambda \pronounced{LAM-duh} & $\psi$, $\Psi$ & psi \pronounced{SIGH, or PSIGH} \\
% 		$\mu$ & mu \pronounced{MEW} & $\omega$, $\Omega$ & omega \pronounced{oh-MAY-guh} \\
% 		\bottomrule
% 	\end{tabular} \\[1.5ex]
% 	Capitals shown are the ones that differ from Roman capitals.
% \end{center}

%----------------------------------------------------------------------------------------
%	GLOSSARY
%----------------------------------------------------------------------------------------

% The glossary needs to be compiled on the command line with 'makeglossaries main' from the template directory

% \newglossaryentry{computer}{
% 	name=computer,
% 	description={is a programmable machine that receives input, stores and manipulates data, and provides output in a useful format}
% }

% % Glossary entries (used in text with e.g. \acrfull{fpsLabel} or \acrshort{fpsLabel})
% \newacronym[longplural={Frames per Second}]{fpsLabel}{FPS}{Frame per Second}
% \newacronym[longplural={Tables of Contents}]{tocLabel}{TOC}{Table of Contents}

% \setglossarystyle{listgroup} % Set the style of the glossary (see https://en.wikibooks.org/wiki/LaTeX/Glossary for a reference)
% \printglossary[title=Special Terms, toctitle=List of Terms] % Output the glossary, 'title' is the chapter heading for the glossary, toctitle is the table of contents heading

%----------------------------------------------------------------------------------------
%	INDEX
%----------------------------------------------------------------------------------------

% The index needs to be compiled on the command line with 'makeindex main' from the template directory

\printindex % Output the index

%----------------------------------------------------------------------------------------
%	BACK COVER
%----------------------------------------------------------------------------------------

% If you have a PDF/image file that you want to use as a back cover, uncomment the following lines

%\clearpage
%\thispagestyle{empty}
%\null%
%\clearpage
%\includepdf{cover-back.pdf}

%----------------------------------------------------------------------------------------

\end{document}
