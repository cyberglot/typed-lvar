\documentclass[main.tex]{subfiles}
\begin{document}

\section{\lambdalvar}
\usingnamespace{lvar}

\todo{What is \lambdalvar?}
\todo{Why is it morally a subset of $\lambda_{\text{LVish}}$, and why does this subset maintain the same strong properties as $\lambda_{\text{LVish}}$}

\begin{figure}
  \noindent
  \begin{mathpar}
    \fbox{$\tm{C}\redc\tm{C'}$}\hfill
    \\
    \inferrule*[lab=E-Put-Err-Incomp]{
      \tm{S(l)} = \tm{\st{d}{\false}}
      \\
      \tm{{d}\sqcupD{d'}} = \tm{\topD}
    }{\tm{\config{\lvarput\;l\;d'}{S}}
      \redc
      \tm{\error}}
    \\
    \inferrule*[lab=E-Put-Err-Freeze]{
      \tm{S(l)} = \tm{\st{d}{\true}}
    }{\tm{\config{\lvarput\;l\;d'}{S}}
      \redc
      \tm{\error}}
  \end{mathpar}
  \caption{Extension to operational semantics for \lambdalvar.}
  \label{fig:lindsey}
\end{figure}
See~\cref{fig:lindsey}.

\paragraph{Operational correspondence}
\begin{theorem}[Simulation]
  For any well-typed $\tm{C}$:
  \begin{itemize}
  \item 
    if $\tm{C}\red\tm{C'}$, then $\tm{\trans{C}}\red\tm{\trans{C'}}$;
  \item
    if $\tm{C}\rede\tm{C'}$, then $\tm{\trans{C}}\rede\tm{\trans{C'}}$.
  \end{itemize}
\end{theorem}
\begin{proof}
  \todo{\typedlambdalvar is a subset of \lambdalvar, and every
    reduction rule in \typedlambdalvar is a reduction rule in \lambdalvar.}
\end{proof}

\begin{theorem}[Reflection]
  For any well-typed $\tm{C}$ in \typedlambdalvar:
  \begin{itemize}
  \item
    if $\tm{\trans{C}}\red\tm{D'}$,
    then $\tm{C}\red\tm{C'}$ and $\tm{\trans{C'}}=\tm{D'}$ for some $\tm{C'}$;
  \item
    if $\tm{\trans{C}}\rede\tm{D'}$,
    then $\tm{C}\rede\tm{C'}$ and $\tm{\trans{C'}}=\tm{D'}$ for some $\tm{C'}$.
  \end{itemize}
\end{theorem}
\begin{proof}
  \todo{Boils down to proving that if the translated term can perform an error
    reductions, then the source term wasn't well-typed.}
\end{proof}

\paragraph{Generalised Independence}
\begin{theorem}[Generalised Independence]
\end{theorem}
\begin{corollary}[Generalised Independence]
\end{corollary}

\paragraph{Determinism}
\begin{lemma}[Strong Confluence, $\red$]
\end{lemma}
\begin{lemma}[Quasi-Confluence, $\rede$]
\end{lemma}
\begin{theorem}[Quasi-Determinism, $\rede$]
\end{theorem}
\begin{corollary}[Determinism, $\rede$]
\end{corollary}

\end{document}

%%% Local Variables:
%%% TeX-master: "main"
%%% End:
