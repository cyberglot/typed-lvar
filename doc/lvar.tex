\documentclass[main.tex]{subfiles}
\begin{document}

\section{LVars: quasi-deterministic parallelism}
\usingnamespace{lvar}

\subsection{The \lambdalvar calculus}
\todo{What is \lambdalvar?}

\lambdalvar is defined over an abstract domain $\mathcal{D}$ which forms a lattice. We write $\botD$ and $\topD$ for the minimum and maximum elements, $\sqsubseteqD$ for its partial order, and $\sqcupD$ for the maximum.

\paragraph*{Terms}
Terms ($\tm{L}$, $\tm{M}$, $\tm{N}$) are defined by the following grammar:
\[
  \begin{array}{lrll}
  \tm{L}, \tm{M}, \tm{N}
  & \coloneqq & \tm{x}
    \sep        \tm{\lambda x.M}
    \sep        \tm{M\;N}                     & \text{functions} \\
  & \sep      & \tm{l}                        
    \sep        \tm{d}
    \sep        \tmJ
    \sep        \tm{K}                        & \text{constants} \\
  & \sep      & \tm{\unit}                    
    \sep        \tm{\letunit{M}{N}}           & \text{units} \\
  & \sep      & \tm{\pair{M}{N}}              
    \sep        \tm{\letpair{x}{y}{M}{N}}     & \text{pairs}
  \\
  \\
  \tm{K}
  & \coloneqq & \tm{\lvarnew}
    \sep        \tm{\lvarfreeze}
    \sep        \tm{\lvarget}
    \sep        \tm{\lvarput}
\end{array}
\]
Let $\tm{x}$, $\tm{y}$, and $\tm{z}$ range over variable names. The term language is the standard $\lambda$-calculus with products and units, extended with LVars~\citep{kuper15}. Let $\tm{l}$ range over location names. Let $\tm{d}$ range over \emph{domain values} from the domain $\mathcal{D}$. Finally, let $\tmJ$ range over \emph{thresholds}. A~threshold $\tmJ$ is a subset of $\mathcal{D}$. A~threshold is pair-wise incompatible, written $\incomp(\tmJ)$, if for all $\tm{d},\tm{d'}\in\mathcal{D}$, if $\tm{d}\neq\tm{d'}$ then $\tm{d}\sqcup\tm{d'}=\tm{\topD}$.

\paragraph*{Configurations}
Configurations ($\tm{C}$), stores ($\tm{S}$), states ($\tm{p}$), and status bits ($\tm{\frz}$) are defined by the following grammar:
\[
\begin{array}{lrll}
  \tm{C}, \tm{D}
  & \coloneqq & \tm{\config{M}{S}}
    \sep        \tm{\error}        & \text{configurations}
  \\
  \tm{S}
  & \coloneqq & \tm{\storeemp}
    \sep        \tm{\storeext{S}{l}{p}} & \text{stores}
  \\
  \tm{p}
  & \coloneqq & \tm{\langle \frz, d \rangle} & \text{states}
  \\
  \tm{\frz}
  & \coloneqq & \tm{\true}
    \sep        \tm{\false}
                                    & \text{status bits}
  \end{array}
\]
A~configuration $\tm{\config{M}{S}}$ consists of a term $\tm{M}$ and a store $\tm{S}$. A~store is a partial function from locations $\tm{l}$ to states $\tm{p}$. A~state consists of a domain value $\tm{d}$ and a status bit $\tm{\frz}$. The status bit records if that particular location has been frozen. We write $\tm{\storeext{S}{l}{p}}$ for the \emph{extension} of a store $\tm{S}$ with state $\tm{p}$ at a \emph{fresh} location $\tm{l}$. We write $\tm{\storeupd{S}{l}{p}}$ for the \emph{update} of a store with state $\tm{p}$ at an \emph{existing} location $\tm{l}$.

We write $\tm{\frz}\sqcupB\tm{\frz'}$ for the maximum on status bits, which is standard Boolean disjunction, and $\tm{\frz}\sqsubseteqB\tm{\frz'}$ for the \emph{total} order on status bits, which is standard Boolean implication.

\paragraph*{Syntactic sugar}
We use syntactic sugar to make terms more readable: we write $\tm{\andthen{M}{N}}$ in place of $\tm{\letunit{M}{N}}$, $\tm{\letbind{x}{M}{N}}$ in place of $\tm{(\lambda x.N)\;M}$, and pattern matching functions $\tm{\lambda\unit.M}$ in place of $\tm{\lambda z.\letunit{z}{M}}$ and $\tm{\lambda\pair{x}{y}.M}$ in place of $\tm{\lambda z.\letpair{x}{y}{z}{M}}$.

\paragraph{Operational semantics}
\label{sec:lvar-semantics}
\begin{mathpar}
  \begin{array}{lrll}
    \tm{V}, \tm{W}
    & \coloneqq & \tm{\lambda x.M}
      \sep        \tm{l}
      \sep        \tm{d}
      \sep        \tmJ
      \sep        \tm{K}
      \sep        \tm{\unit}
      \sep        \tm{\pair{V}{W}}
    \\
    \tm{E}
    & \coloneqq & \tm{\hole}
      \sep        \tm{M\; E}
      \sep        \tm{E\; M}
      \sep        \tm{(M, E)}
      \sep        \tm{(E, M)} \\
    & \sep      & \tm{\letunit{E}{M}}
      \sep        \tm{\letpair{x}{y}{E}{M}}
  \end{array}
\end{mathpar}

\begin{figure*}
  \noindent
  \begin{minipage}[t]{.75\textwidth}
    \fbox{$\tm{C}\red\tm{C'}$}\hfill
    \\
    \begin{mathpar}
      \begin{array}{llrl}
        \LabTirName{E-Lam}     & \tm{\config{(\lambda x.M)\;V}{S}}
                               & \red & \tm{\config{\subst{M}{V}{x}}{S}}
        \\
        \LabTirName{E-Unit}    & \tm{\config{\letunit{\unit}{M}}{S}}
                               & \red & \tm{\config{M}{S}}
        \\
        \LabTirName{E-Pair}    & \tm{\config{\letpair{x}{y}{\pair{V}{W}}{M}}{S}}
                               & \red & \tm{\config{\subst{\subst{M}{V}{x}}{W}{y}}{S}}
      \end{array}
      \\
      \inferrule*[lab=E-New]{
        \tm{l}\not\in\dom(\tm{S})
      }{\tm{\config{\lvarnew}{S}}
        \red
        \tm{\config{l}{S, l \mapsto\st{\botD}{\false}}}}

      \inferrule*[lab=E-Get]{
        \tm{S(l)} = \tm{\st{d}{\frz}}
        \\
        \incomp({\tmJ})
        \\
        \tm{d'}\in\tmJ
        \\
        \tm{d'}\sqsubseteqD\tm{d}
      }{\tm{\config{\lvarget\;l\;\tmJ}{S}}
        \red
        \tm{\config{d'}{S}}}

      \inferrule*[lab=E-Put]{
        \tm{S(l)} = \tm{\st{d}{\false}}
        \\
        \tm{{d}\sqcupD{d'}} \neq \tm{\topD}
      }{\tm{\config{\lvarput\;l\;d'}{S}}
        \red
        \tm{\config{()}{\storeupd{S}{l}{\st{d'}{\false}}}}}
            
      \inferrule*[lab=E-Freeze]{
        \tm{S(l)} = \tm{\st{d}{\frz}}
      }{\tm{\config{\lvarfreeze}{S}}
        \red
        \tm{\config{d}{\storeupd{S}{l}{\st{d}{\true}}}}}
    \end{mathpar}
  \end{minipage}%
  \begin{minipage}[t]{.25\textwidth}
    \fbox{$\tm{C}\rede\tm{C'}$}\hfill
    \\
    \begin{mathpar}
      \inferrule*[lab=E-Lift]{
        \tm{\config{M}{S}} \red \tm{\config{N}{S'}}
      }{\tm{\config{\plug{E}{M}}{S}}
        \rede
        \tm{\config{\plug{E}{N}}{S'}}}
    \end{mathpar}
  \end{minipage}
  \caption{Operational semantics for \typedlambdalvar.}
  \label{fig:semantics}
\end{figure*}
\begin{figure}
  \noindent
  \begin{mathpar}
    \fbox{$\tm{C}\redc\tm{C'}$}\hfill
    \\
    \inferrule*[lab=E-Put-Err-Incomp]{
      \tm{S(l)} = \tm{\st{d}{\false}}
      \\
      \tm{{d}\sqcupD{d'}} = \tm{\topD}
    }{\tm{\config{\lvarput\;l\;d'}{S}}
      \redc
      \tm{\error}}
    \\
    \inferrule*[lab=E-Put-Err-Freeze]{
      \tm{S(l)} = \tm{\st{d}{\true}}
    }{\tm{\config{\lvarput\;l\;d'}{S}}
      \redc
      \tm{\error}}
  \end{mathpar}
  \caption{Extension to operational semantics for \lambdalvar.}
  \label{fig:lindsey}
\end{figure}
See~\cref{fig:semantics,fig:lindsey}.

\subsection{Relation to \lambdalvish}
\todo{%
  Why is \lambdalvar morally a subset of $\lambdalvish$, and why does this subset maintain the same strong properties as $\lambdalvish$?}

\subsection{Metatheory}
\begin{theorem}[Generalised Independence]
\end{theorem}
\begin{lemma}[Strong Confluence, $\redc$]
\end{lemma}
\begin{lemma}[Quasi-Confluence, $\rede$]
\end{lemma}
\begin{theorem}[Quasi-Determinism, $\rede$]
\end{theorem}

\end{document}

%%% Local Variables:
%%% TeX-master: "main"
%%% End:
